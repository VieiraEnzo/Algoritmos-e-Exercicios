\documentclass[10pt, twocolumn]{article}
\usepackage[utf8]{inputenc}
\usepackage[english]{babel}
\usepackage[utf8]{inputenc}
\usepackage{lscape}
\usepackage[T1]{fontenc}
\usepackage{fancyhdr}
\usepackage{rotating}
\usepackage{ragged2e}
\usepackage[a4paper, landscape, margin=0.8in]{geometry}
\usepackage{blindtext}
\usepackage{amsmath}
\usepackage{relsize}
\usepackage{graphicx}
\usepackage{float}
\usepackage{placeins}
\usepackage{enumitem}
\usepackage{booktabs}
\usepackage{array}
\usepackage{xcolor,colortbl}
\usepackage{tabu}
\usepackage{url}
\usepackage{textcomp}
\usepackage{diagbox}
\usepackage{caption}
\usepackage{amsfonts}
\usepackage{amsmath}

\usepackage{hyperref}
\hypersetup{
    colorlinks=true,
    linkcolor=blue,
    filecolor=magenta,      
    urlcolor=blue,
}

%color celula da tabela
\usepackage{xcolor,colortbl}

\setlength{\columnseprule}{0.4pt}
\setlength{\columnsep}{3em}
\graphicspath{ {img/} }
\restylefloat{table}

\newcommand{\mc}[2]{\multicolumn{#1}{c}{#2}}
\definecolor{Gray}{gray}{0.85}
\newcolumntype{a}{>{\columncolor{Gray}}c}

\newcommand{\stirlingfirst}[2]{\genfrac{[}{]}{0pt}{}{#1}{#2}}
\newcommand{\stirlingsecond}[2]{\genfrac{\{}{\}}{0pt}{}{#1}{#2}}

\pagestyle{fancy}

\begin{document}

\rhead{UFRJ}
\lhead{É o shadow cara}
\begin{flushleft}

% \chapter*{Matematica}
\section{Identidades}

\subsection{Series}

$$\sum_{i=1}^{n} i^{2} = \frac{n(n+1)(2n+1)}{6}  \qquad  \sum_{i=1}^{n} i^{3} = \frac{n^{2}(n+1)^{2}}{4} = \left(\sum_{i=1}^n i\right)^2$$

$$ g_k(n) = \sum_{i=1}^n i^k = \frac{1}{k+1} \left( n^{k+1} + \sum_{j=1}^k \binom{k+1}{j+1} (-1)^{j+1} g_{k-j}(n) \right) $$

$$\sum_{i=0}^{n} ic^{i} = \frac{nc^{n+2} - (n+1)c^{n+1} + c}{(c-1)^{2}}, \quad c \neq 1$$

$$\sum_{i=0}^{\infty} ic^{i} = \frac{c}{(1-c)^{2}}, \quad |c| < 1$$

\subsection{Binomial Identities}

\bigskip

$$
\begin{array}{cc}
\binom{n}{k} = \frac{n}{k}\binom{n-1}{k-1}
&
\binom{n-1}{k} - \binom{n-1}{k-1} = \frac{n - 2k}{k} \binom{n}{k}
\\[10pt]
\binom{n}{h}\binom{n-h}{k} = \binom{n}{k}\binom{n-k}{h} &
\binom{n}{k} = \frac{n+1-k}{k} \binom{n}{k-1}
\\[10pt]
\sum_{k = 0}^n k\binom{n}{k} = n 2^{n-1} &
\sum_{k = 0}^n k^2 \binom{n}{k} = (n + n^2)2^{n-2}
\\[10pt]
\sum_{j = 0}^k\binom{m}{j} \binom{n-m}{k-j} = \binom{n}{k} &
\sum_{j = 0}^m \binom{m}{j}^2 = \binom{2m}{m}
\\[10pt]
\sum_{m = 0}^n \binom{m}{j} \binom{n-m}{k-j} = \binom{n+1}{k+1} &
\sum_{m = k}^n \binom{m}{k} = \binom{n+1}{k+1}
\\[10pt]
\sum_{r = 0}^m \binom{n+r}{r} = \binom{n+m+1}{m} &
\sum_{k=0}^{\lfloor n/2 \rfloor} \binom{n-k}{k} = \text{Fib}(n+1)
\\[10pt]
(x + y)^{n} = \sum_{k=0}^{n} \binom{n}{k} x^{n-k} y^{k} &
(1 + x)^{n} = \sum_{k=0}^{n} \binom{n}{k} x^{k}
\\[10pt]
2\sum_{i = L}^R \binom{n}{i} - \binom{n}{L} - \binom{n}{R} = \sum_{i = L+1}^R \binom{n+1}{i}
\end{array}
$$

\section{Number Theory}

\subsection{Identities}

$$ \sum_{d|n} \varphi(d) = n$$

$$ \sum_{\substack{i < n \\ \text{gcd}(i, n) = 1}} i = n \cdot \frac{\varphi(n)}{2} $$

$$ |\{(x, y) : \, 1 \leq x, y \leq n, \, \text{gcd}(x, y) = 1\}| = \sum_{d = 1}^n \mu(d) \left \lfloor \frac{n}{d} \right \rfloor^2$$

$$\sum_{x = 1}^n \sum_{y = 1}^n \text{gcd}(x, y)
= \sum_{k = 1}^n k \sum_{k|l}^{n} \left \lfloor \frac{n}{l} \right \rfloor^2 \mu\left(\frac{l}{k}\right)$$
    
$$\sum_{x = 1}^n \sum_{y = x}^n \text{gcd}(x, y) = \sum_{g = 1}^n \sum_{g|d}^n g \cdot \varphi\left(\frac{d}{g}\right)$$

$$\sum_{x = 1}^n \sum_{y = 1}^n \text{lcm}(x, y)
= \sum_{d = 1}^n d \, \mu(d) \sum_{d|l}^{n} l \binom{\lfloor \frac{n}{l} \rfloor + 1}{2}^2$$

$$\sum_{x = 1}^n \sum_{y = x+1}^n \text{lcm}(x, y) = \sum_{g = 1}^n \sum_{g|d}^n d \cdot \varphi\left(\frac{d}{g}\right) \cdot \frac{d}{g} \cdot \frac 1 2$$

$$\sum_{x \in A} \sum_{y \in A} \text{gcd}(x, y)
= \sum_{t = 1}^n \left(\sum_{l | t} \frac{t}{l} \mu(l)\right) \left(\sum_{t|a} \text{freq}[a]\right)^2$$

$$\sum_{x \in A} \sum_{y \in A} \text{lcm}(x, y)
= \sum_{t = 1}^n \left(\sum_{l | t} \frac{l}{t} \mu(l)\right) \left(\sum_{a \in A, \, t|a} a\right)^2$$

\subsection{Large Prime Gaps}
For numbers until $10^9$ the largest gap is 400.\\
For numbers until $10^{18}$ the largest gap is 1500.\\[0.5cm]

\subsection{Prime counting function - \texorpdfstring{$\pi(x)$}{}} The prime counting function is asymptotic to $\frac{x}{\log x}$, by the prime number theorem.

\ 

\begin{tabular}{|c|c|c|c|c|c|c|c|c|}
\hline
  \cellcolor{gray!40} x&10&$10^2$&$10^3$&$10^4$&$10^5$&$10^6$&$10^7$&$10^8$\\ \hline
  \cellcolor{gray!40} $\pi(x)$& 4 & 25 & 168 & 1\,229 & 9\,592 & 78\,498 & 664\,579 & 5\,761\,455\\ \hline
\end{tabular}

\ 

\subsection{Some Primes}

999999937 $\quad$ 1000000007 $\quad$ 1000000009 $\quad$ 1000000021 $\quad$ 1000000033
$10^{18} - 11 \quad\quad 10^{18} + 3 \quad\quad\quad 2305843009213693951 = 2^{61} - 1$

\subsection{Number of Divisors}

\begin{table}[H]
    \centering
    \begin{tabular}{|c|c|c|c|c|c|c|c|c|c|c|c|c|}
        \hline
        \cellcolor{gray!40} $n$ & 6 & 60 & 360 & 5040 & 55440 & 720720 & 4324320 & 21621600 \\
        \hline
        \cellcolor{gray!40} $d(n)$ & 4 & 12 & 24 & 60 & 120 & 240 & 384 & 576 \\
        \hline
    \end{tabular}
\end{table}

\vspace{-30pt}

%  367567200 d(n) = 1152
%  6983776800 d(n) = 2304
%  13967553600 d(n) = 2688
%  321253732800 d(n) = 5376
%  18401055938125660800 d(n) = 184320

\begin{table}[H]
    \centering
    \begin{tabular}{|c|c|c|c|c|}
        \hline
        \cellcolor{gray!40} $n$ & 367567200 & 6983776800 & 13967553600 & 321253732800
        \\
        \hline
        \cellcolor{gray!40} $d(n)$ & 1152 & 2304 &  2688 &
        5376 \\
        \hline
    \end{tabular}
\end{table}

18401055938125660800 $\approx 2\text{e}18$ is   highly composite with 184320 divisors.

For numbers up to $10^{88}$, $d(n) < 3.6 \sqrt[3]{n}$.

\subsection{Lucas's Theorem}

$$ \binom{n}{m} \equiv \prod_{i=0}^k \binom{n_i}{m_i} \quad (\text{mod } p) $$ 

For $p$ prime. $n_i$ and $m_i$ are the coefficients of the representations of $n$    and $m$ in base $p$. In particular, $\binom{n}{m}$ is odd if and only if $n$ is a submask of $m$.

\subsection{Fermat's Theorems}
Let $p$ be a prime number and $a$ an integer, then:
$$a^p \equiv a \quad (\text{mod } p)$$
$$a^{p-1} \equiv 1 \quad (\text{mod } p)$$

\textbf{Lemma:} Let $p$ be a prime number and $a$ and $b$ integers, then: 
$$(a+b)^{p} \equiv a^{p} + b^{p} \quad (\text{mod } p)$$

\textbf{Lemma:} Let $p$ be a prime number and $a$ an integer. The inverse of $a$ modulo $p$ is $a^{p-2}$:

$$a^{-1} \equiv a^{p-2} \quad (\text{mod } p)$$

\subsection{Taking modulo at the exponent}

If $a$ and $m$ are coprime, then

$$ a^n \equiv a^{n \text{ mod } \varphi(m)} \quad (\text{mod } m) $$

Generally, if $n \geq \log_2 m$, then

$$ a^n \equiv a^{\varphi(m) + [n \text{ mod } \varphi(m)]} \quad (\text{mod } m) $$

\subsection{Mobius invertion}

If $g(n) = \sum_{d|n} f(d)$, then $f(n) = \sum_{d|n} g(d) \mu(\frac{n}{d})$.

A more useful definition is: $\sum_{d|n} \mu(d) = [n = 1]$

Example, sum of LCM:

\begin{align*}
    \sum_{i = 1}^n \sum_{j = 1}^n \text{lcm}(i, j) &=
    \sum_{k = 1}^n\sum_{i=1}^n\sum_{j=1}^n [\text{gcd}(i, j) = k] \frac{ij}{k} \\
    &= \sum_{k = 1}^n\sum_{a=1}^{\lfloor \frac{n}{k} \rfloor}\sum_{b=1}^{\lfloor \frac{n}{k} \rfloor} [\text{gcd}(a, b) = 1] abk \\
    &= \sum_{k = 1}^n k \sum_{a=1}^{\lfloor \frac{n}{k} \rfloor} a \sum_{b=1}^{\lfloor \frac{n}{k} \rfloor} b \sum_{d=1}^{\lfloor \frac{n}{k} \rfloor}[d|a]\,[d|b] \, \mu(d) \\
    &= \sum_{k = 1}^n k \sum_{d=1}^{\lfloor \frac{n}{k} \rfloor}  \mu(d) \left(\sum_{a=1}^{\lfloor \frac{n}{k} \rfloor} [d|a] \, a\right) \left(\sum_{b=1}^{\lfloor \frac{n}{k} \rfloor} [d|b] \, b\right) \\
    &= \sum_{k = 1}^n k \sum_{d=1}^{\lfloor \frac{n}{k} \rfloor}  \mu(d) \left(\sum_{p=1}^{\lfloor \frac{n}{kd} \rfloor} p\right) \left(\sum_{q=1}^{\lfloor \frac{n}{kd} \rfloor} q\right) \\
\end{align*}

\subsection{Chicken McNugget Theorem}

Given two \textbf{coprime} numbers $n, m$, the largest number that cannot be written as a linear combination of them is $nm - n - m$.

\begin{itemize}
    \item There are $\frac{(n-1)(m-1)}{2}$ non-negative integers that cannot be written as a linear combination of $n$ and $m$;
    \item For each pair $(k, nm - n - m - k)$, for $k \geq 0$, exactly one can be written.
\end{itemize}


\subsection{Harmonic Lemma}

This technique computes sums of the form $\sum_{i=1}^{n} f\left(\left\lfloor \frac{n}{i} \right\rfloor\right)$ in $O(\sqrt{n})$.

The value of $\lfloor \frac{n}{i} \rfloor$ is constant over blocks. If we are at index $l$, the value $v = \lfloor \frac{n}{l} \rfloor$ remains constant up to index $r = \lfloor \frac{n}{v} \rfloor$. We can iterate through these blocks instead of individual indices.

\begin{verbatim}
long long sum = 0;
for (int l = 1, r; l <= n; l = r + 1) {
    int val = n / l;
    r = n / val;
    sum += (long long)(r - l + 1) * f(val);
}
\end{verbatim}

\textbf{Generalization:} For sums like $\sum_{i=1}^{n} g(i) \cdot f(\lfloor \frac{n}{i} \rfloor)$, the contribution of a block $[l, r]$ is:
$$ (G(r) - G(l-1)) \cdot f\left(\left\lfloor \frac{n}{l} \right\rfloor\right) $$
where $G(k)$ is the prefix sum $\sum_{i=1}^k g(i)$, which must be efficiently computable.
\section{Geometry}
    
\subsection{Pythagorean Triples}

For all natural $a, b, c$ satisfying $a^2 + b^2 = c^2$ there exist $m, n \in \mathbb{N}$ and $m > n$ such that (reverse is also true):
$$a = m^2 - n^2 \qquad b = 2mn \qquad c = m^2 + n^2$$

\subsection{Heron's Formula}

The area of a triangle can be written as $A = \sqrt{s\,(s-a)\,(s-b)\,(s-c)}$, where $a, b, c$ are the lengths of its sides and $s = \frac{a+b+c}{2}$.

This can be generalized to compute the area $A$ of a quadrilateral with sides $a, b, c, d$, with $s = \frac{a+b+c+d}{2}$ and $\alpha, \gamma$ any two opposite angles:

$$ A = \sqrt{(s-a)(s-b)(s-c)(s-d) - abcd\left( \cos ^2 \left( \frac{\alpha+\gamma}{2} \right) \right)} $$

\subsection{Pick's Theorem}

The area of a simple polygon whose vertices have integer coordinates is:

\[
A = I + \frac{B}{2} - 1
\]

where $I$ is the number of interior integer points, and $B$ is the number of integer points in the border of the polygon.

\subsection{Colinear Points}

Three points are colinear on $\mathbb{R}^2$ iff:

$$ \begin{vmatrix}
x_A & y_A & 1 \\
x_B & y_B & 1 \\
x_C & y_C & 1 \\
\end{vmatrix}  = 0 $$

The absolute value of this determinant is twice the area of the triangle $ABC$.

\subsection{Coplanar Points}

Four points are coplanar in $\mathbb{R}^3$ iff:

$$ \begin{vmatrix}
x_A & y_A & z_A & 1 \\
x_B & y_B & z_B & 1 \\
x_C & y_C & z_C & 1 \\
x_D & y_D & z_D & 1 \\
\end{vmatrix}  = 0 $$

\subsection{Trigonometry}

\subsubsection{Angle Sum}

$$ \sin(a \pm b) = \sin a \cos b \pm \cos \ \sin b $$
$$ \cos(a \pm b) = \cos a \cos b \mp \sin a \sin b $$
$$ \tan (a \pm b) = \frac{\tan a \pm \tan b}{1 \mp \tan a \tan b}$$

\subsubsection{Sum-to-Product Transformation}

$$ \sin a \pm \sin b = 2 \sin\frac{a \pm b}{2} \cos\frac{a \mp b}{2} $$
$$ \cos a + \cos b = 2 \cos\frac{a+b}{2} \cos\frac{a-b}{2} $$
$$ \cos a - \cos b = -2 \sin\frac{a+b}{2} \sin\frac{a-b}{2} $$
$$ \tan a \pm \tan b = \frac{\sin(a \pm b)}{\cos a \cos b} $$

\subsection{Centroid of a polygon}

The coordites of the centroid of a non-self-intersecting closed polygon is:

$$ \frac{1}{3A} \left(\sum_{i = 0}^{n-1}(x_i+x_{i+1})(x_iy_{i+1} - x_{i+1}y_i) , \sum_{i = 0}^{n-1}(y_i+y_{i+1})(x_iy_{i+1} - x_{i+1}y_i) \right), $$

where $A$ is twice the signed area of the polygon.

\section{Other}

\subsubsection{String Matching with Wildcards}


Consider a text $T$ and a pattern $P$. $P$ and $T$ may have wildcards that will match any character. The problem is to get the positions where $P$ occur in $T$.

If we define the value of the characters such that the wildcard is zero and the other characters are positive, there is a matching at position $i$ iff $\sum_{j=0}^{|P|-1}P[j]T[i+j](P[j] - T[i + j])^2 = 0$. Then, one can evaluate each term of 
$$ \sum_{j=0}^{|P|-1}(P[j]^3T[i+j] - 2P[j]^2T[i+j]^2 + P[j]T[i + j]^3) $$ 

using three convolutions.

\end{flushleft}

\end{document}              